\chapter{Introduction}

In just above 3 years, this community have submitted 10,000 puzzles\footnote{Actually, it's a bit 
more than that --- there were some puzzle packs in the archive, but this pack is the 10,000th entry 
to the archive.} to the CTC Discord Archive! That's a mind boggling number of puzzles. 

To celebrate, a whole bunch of setters have come together to create a pack of \textbf{100 puzzles}! 
They come in all kinds of genres, sizes, and difficulties, so there should be something for everyone.
When we first started this project, we didn't know yet whether we'd be able to pull it off. But in
the end, it went relatively smoothly. But was it a \emph{good} pack? We'll let you be the judge.

% This document has divided the puzzles into several categories: 
% \begin{enumerate}
%   \item \emph{Sudoku and Latin Square.} 
%   \item \emph{Loop and Path.} Basically, puzzles where your goal is to draw a line! This includes 
%       genres like Masyu, Simple Loop, or Slitherlink.
%   \item \emph{Object Placement.} 
%   \item \emph{Region Division.} 
%   \item \emph{Shading.} 
%   \item \emph{Others.}
% \end{enumerate}
% Sudoku is, of course, the thing that the CTC YouTube channel and this server is most known for. 
% It's also the largest chapter in the pack by puzzle count, which is why I decided to make it the 
% first chapter.

All of the puzzles in this pack are independent of each other, unless explicitly mentioned (in DiMono's
four puzzles, specifically), so you may solve as few/many as you want, and in any order. Here are some ideas:
\begin{itemize}
  \item Do a 100\% completion of the pack.
  \item Print it out and forget about it.
  \item Solve exactly 1 puzzle in the whole pack.
  \item Decide that you don't like Sudoku, and solve everything except for Chapter 1.
  \item Roll a d100 and solve the puzzle that corresponds to the number rolled, then repeat until satisfied.
\end{itemize}

Thank you to everyone in the server who has made this community as awesome as it is, and thank you to Mark 
and Simon for bringing us together. Here's to another 10,000 great puzzles!

\vspace{2em}

\hfill \emph{- Lavaloid}
