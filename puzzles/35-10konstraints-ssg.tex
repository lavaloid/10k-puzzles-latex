\section[10Konstraints | SSG {[\emph{Multi-Variant Sudoku}]}]{10Konstraints | {\normalfont SSG}}
\label{sec:35-10konstraints-ssg}
\puzzleinfo{Multi-Variant Sudoku}{3.5}

\puzzleimage{./puzzle_images/35-10konstraints-ssg}
\subsection*{Rules}
\begin{markdown}
Normal sudoku rules apply.



**Arrow**: Digits along an arrow must sum to the digit in the connected circle.



**Consecutive Pairs**: Cells connected by a white dot must contain consecutive digits.



**Entropic Lines**: Along a beige line any run of three cells must contain one low {1,2,3}, one medium {4,5,6}, and one high {7,8,9} digit.



**German Whispers**: Successive digits along a green line must differ by at least 5.



**Ratio Pairs**: Cells connected by a black dot must contain digits in a ratio of 1:2.



**Renban**: Each purple line must contain a non-repeating set of consecutive digits which may appear in any order.



**Sandwich**: Uncircled clues outside the grid give the sum of the digits placed between the 1 and 9 in that row or column.



**Thermo**: Digits along a thermometer must strictly increase starting at the bulb.



**X Pairs**: Cells connected by an X must contain digits summing to 10.



**X-Sums**: Circled clues outside the grid give the sum of the first X digits from the position of the clue, where X is the first digit encountered.
\end{markdown}
\subsection*{Links}
\begin{tabularx}{\textwidth}{l X}
\emph{F-puzzles} & \url{https://f-puzzles.com/?id=2gaqeurn} \\
\emph{CTC App} & \url{https://tinyurl.com/2ae6v74y} \\
\end{tabularx}
\pagebreak
