\section{The Abominat10000n | {\normalfont MicroStudy, ymmi}}
\label{sec:42-the-abominat10000n-microstudy-ymmi}
\puzzleinfo{Nikoji, Curve Data, Gemini Loop, Dominion, Pentominous}{2.5}
Do not ask why this was created. It will grow even stronger if you do.
\puzzleimage[1]{./puzzle_images/42-the-abominat10000n-microstudy-ymmi}
\subsection*{Rules}
\begin{markdown}
Top Left: NIKOJI

Top Right: Curve Data

Center: Gemini Loop

Bottom Left: Dominion

Bottom Right: Pentominous



**There are no numerals in the grid. The "10000" and "10K" are made up of the letters I and O and are to be treated as such.**



Detailed Rules:

- NIKOJI: Divide the grid into regions of orthogonally connected cells, each containing exactly one clue. Regions with the same clue type must be exactly identical in shape, orientation, and position relative to the clue. Regions with different clue types may not be the same shape, counting rotations and reflections as the same.

- Curve Data: Draw lines between the centers of cells so that each connected figure goes through exactly one clue, and all cells are used by a figure. Clues show how their figures turn and connect with themselves, not allowing rotation or reflection. The length of each line segment can be expanded or reduced, as long as it is at least 1.

- Gemini Loop: Draw a non-intersecting loop through the centers of all cells. Cells containing the same letter must be entered by the loop from the same directions. Cells containing different letters must not.

- Dominion: Shade some dominoes of cells to divide the grid into unshaded areas. Shaded dominoes may not touch orthogonally. Clues cannot be shaded, and each orthogonally connected area of unshaded cells contains exactly one type of clue, and all instances of it.

- Pentominous: Divide the grid into regions of five orthogonally connected cells so that no two regions of the same shape share an edge, counting rotations and reflections as the same. Clued cells must belong to a region with the pentomino shape associated with that letter.
\end{markdown}
\subsection*{Links}
\begin{tabularx}{\textwidth}{l X}
\emph{Penpa+ (Main Grid)} & \url{https://tinyurl.com/49tbwdcy} \\
\emph{Penpa+ (NIKOJI)} & \url{https://tinyurl.com/2h6buac3} \\
\emph{Puzz.link (Curve Data)} & \url{https://tinyurl.com/ec43xb88} \\
\emph{Penpa+ (Gemini Loop)} & \url{https://tinyurl.com/5e9vyxkp} \\
\emph{Penpa+ (Dominion)} & \url{https://tinyurl.com/4rm3cnd4} \\
\emph{Penpa+ (Pentominous)} & \url{https://tinyurl.com/25juwust} \\
\end{tabularx}
\pagebreak
